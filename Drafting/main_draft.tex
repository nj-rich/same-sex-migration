\documentclass[12pt,letterpaper]{article}
\usepackage{datetime} %maybe don't need
\usepackage{amsmath}
\usepackage{amsfonts}
\usepackage{amssymb}
\usepackage{graphicx}
\usepackage{natbib}
\usepackage{url}
\usepackage{setspace}
\usepackage{geometry}

%I added the following
\usepackage[colorlinks=true, linkcolor=blue, citecolor=blue, urlcolor=blue]{hyperref} %for text hyperlinks
\usepackage{graphicx}
\usepackage{longtable} %for Gerstmann
\usepackage{booktabs}  %for pretty tables
\usepackage{lscape} %forget where horizontal comes from
\usepackage{rotating} %forget where horizontal comes from
\usepackage{tabularx}  %to help format tables
\usepackage{caption} %to modify caption
\usepackage{placeins} %for FloatBarrier
%removes colon after number label
\captionsetup[figure]{labelsep=space} %might remove if want to label figures differently

% Adjust margins to fit the AER style
\geometry{top=1in, bottom=1in, left=1in, right=1in}

% Title setup
\title{Gay Rights and Interstate Migration: Have People Changed Since 2015?\footnote{Acknowledgements: I am indebted to Dr. Reynoso, Dr. Mueller-Smith, Dr. Montgomery, and Dr. Dominguez for their invaluable guidance. I also appreciate all my friends and fellow economics honors students who read drafts and offered incredibly helpful advice. I also have to thank Eugenia Quintanilla for starting my social science research journey.}}
\author{Noah Rich\footnote{University of Michigan, njrich@umich.edu.}}
\date{\today}

% Start the document
\begin{document}

\maketitle


% Abstract section
\begin{abstract}
Same-sex marriage was federally legalized in the US in 2015. This paper investigates to what extent migration patterns of individuals in same-sex and opposite-sex relationships converged after 2015. It does this by comparing migration patterns to states that locally legalized same-sex marriage and those that did not. It offers a novel analysis of federal marriage legalization and introduces the idea of a federal dilution effect, or the idea that federal policy change can dampen the effects of previously enacted state policy. Using a triple difference design and data from the American Community Survey from 2011 through 2019, I find no statistically significant difference between where individuals in same-sex and opposite-sex relationships migrate before and after 2015. I suggest that this could be due to data quality issues and a weak relationship between same-sex marriage legalization and migration. Future research might try to understand the “dilution effect” in other contexts with different tools.
\end{abstract}

\newpage

% Main content section
\section{Introduction}

In 2015, same-sex marriage was federally legalized across the United States. This marked a landmark moment in the LGBTQ+ civil rights movement: same-sex relationships were now afforded the same legal rights as opposite-sex relationships across the country. Before 2015, some states had already legalized same-sex marriage. However, this left a patchwork of rules and regulations across the US for individuals in same-sex relationships. 2015 is important because it unified this patchwork.

This then leads to the question, how did individuals in same-sex relationships respond to the geographic expansion of their rights? Might they feel more comfortable moving to places where they previously had fewer rights? In this paper, I investigate to what extent migration patterns of individuals in same-sex and opposite-sex relationships converged after 2015. I do this by using a triple difference regression on American Community Survey (ACS) data from 2011 through 2019 to compare migration patterns to states that locally legalized same-sex marriage and those that did not. 

Researchers have investigated how heterogenous legal and social treatment have impacted individuals in same-sex relationships relative to those in opposite-sex relationships. Generally, individuals in same-sex and opposite-sex relationships have similar behavior and outcomes when treated equivalently, and different behaviors and outcomes when treated differently \citep{2}. For example, individuals in same-sex relationships were more likely than those in opposite-sex relationships to move to states with legal same-sex marriage \citep{1, 12}.

This also makes sense considering how economics traditionally understand migration. Economic theory predicts that an individual moves when the utility of a different location is larger than the utility of an individual’s current location and the costs of moving \citep{8, 12}. Before 2015, same-sex marriage legalization produced utility differentials across states.

Notably, there is less research on how federal legalization impacted state-by-state migration. This paper fills that gap. It also introduces the idea that federal legalization could “dilute” the effect of “local legalization.” Both state legalization of same-sex marriage and federal legalization of same-sex marriage are examples of legalization. However, while “local legalization” might give individuals in same-sex relationships an additional incentive to move to a state, federal legalization might “dilute” the effects of that benefit by giving all states, those that “locally legalized” and those that did not, that same potential benefit. 

Thus, under this framework, I expect the “local legalization” of same-sex marriage to be \textit{less} of a migration “pull factor” to a state after 2015. I also expect that having “not-locally-legalized” same-sex marriage to be \textit{less} of a “push factor” out of a state after 2015.

As stated above, I test for these effects by performing a triple differences regression on ACS data from 2011 through 2019. In this data, I observe whether an individual is in a same-sex relationship and if they lived in a different state a year prior. I also use data from \citet{1} to identify states that “locally legalized” same-sex marriage and from (ANES)  to identify the level of popular support individuals in same-sex relationships have across states.

I find statistically insignificant differences between where individuals in same-sex and opposite-sex relationships migrate before and after 2015. This is robust to several variations of my model. I suspect I get these results because of relatively poor data quality and because the relationship between same-sex marriage and migration is relatively weak. While the ACS has a large sample, the fraction that are identified as in same-sex relationships or as migrants is relatively small. While same-sex marriage legalization might have some impact on individuals’ migration decisions, it might matter more in other contexts. For example, I find a marked increase in identified individuals in same-sex relationships after 2015. 

The rest of the paper is organized as follows: section 2 addresses the literature on individuals in same-sex relationships and migration, section 3 describes the theoretical and empirical models used in this paper, section 4 describes data sources, section 5 discusses results, and section 6 concludes.

\section{Literature Review}
\subsection{Same-Sex Marriage}
Economists have been interested in marriage since at least the 1970s. In a landmark series of papers, \citet{9} described how marriage can be understood in a traditional cost-benefit choice utility framework. Specifically, two people enter a marriage if, and only if, it increases both of their lifetime utility. Many unique benefits can be ascribed to marriage in the US. Legally, there are at least 1,138 benefits listed in federal law, including rights related to citizenship, taxes, benefits, healthcare, and parenthood \citep{1, 8}. Other intangible benefits include personal meaning and social recognition, status, and acceptance \citep{8}. On the other hand, costs include legal fees and the intangible costs of finding a partner and making the choice to commit to marriage \citep{9}. 

Over the past 30 years, the legal landscape for individuals in same-sex relationships has changed dramatically in the United States. In 1996, Congress passed the Defense of Marriage Act (DOMA), which defined marriage as being strictly between a man and a woman \citep{5}. Seven short years later, Massachusetts became the first state to legalize same-sex marriage \citep{1, 3, 5}. Between then and 2015, more than half of all U.S. states followed \citep{12}. In 2013, the Supreme Court invalidated DOMA’s definition of marriage, and in 2015, outright legalized same-sex marriage \citep{1, 3, 5, 12}. These legal changes have led to changes in behavior that economists have studied.

\subsection{LGBTQ+ Economics}

As stated in the introduction, differences between individuals in same-sex and opposite-sex relationships often occur because of outside social and legal factors, unrelated to sexuality \citep{2}. These differences can be found in domains including education, the likelihood of forming stable relationships, income, the likelihood of having children, and geographic distribution \citep{2, 11, 6, 7, 8, 10, 7}. For example, gay men and lesbian women tend to be more educated than heterosexual people \citep{2, 7, 11}. Same-sex cohabiting relationships tend to form at a lower rate than opposite-sex cohabiting relationships; however, same-sex relationships tend to be less homogenous across age and racial lines \citep{2, 7}. Gay men tend to earn less than straight men, while lesbian women tend to earn more than straight women \citep{6}. Those in same-sex relationships face higher costs to raising children, and tend to have children at lower rates than those in opposite-sex relationships \citep{8, 10, 11}. With more disposable income, this then in part leads to gay men being more likely to live in expensive, high-amenity places like San Francisco relative to straight men \citep{7, 10, 11, 13}. 

Many of these differences are specifically related to those in same-sex relationships being denied access to marriage. Because of this, researchers have specifically studied how marriage legalization has decreased some heterogeneity between individuals in same-sex and opposite-sex relationships. For example,\citet{3} found some evidence that same-sex marriage legalization has led to an increase in employment of individuals in same-sex relationships relative to individuals in opposite-sex relationships. With the tax and healthcare benefits and adoption rights afforded by legal marriage, same-sex spouses began to divide work more like opposite-sex spouses, with one partner more likely to specialize in paid employment and the other more likely to specialize in unpaid housework and childcare \citep{3, 4, 6}. Further, legal same-sex marriage brought a degree of social acceptance: there is some evidence that same-sex marriage legalization led to more popular support for same-sex marriage \citep{3, 21}. 

\subsection{LGBTQ+ Economics and Migration}
The revealed importance of same-sex marriage legalization for individuals in same-sex relationships, discussed above, suggests that it might be something that individuals would be willing to move for. Below, I discuss how economists understand migration, and how it connects to same-sex marriage legalization.

Like marriage, economists also understand migration in a traditional utility framework \citep{12, 8}. When the utility of a different state is greater than the utility of one’s current state and costs of moving, an individual might want to move to a new state. Monetary costs and benefits include state-related income differences and moving costs \citep{1, 15, 16, 17}. Non-monetary costs and benefits might include distance from family, place attachment, and general satisfaction \citep{1, 15}. Preferences vary across different subpopulations, so it makes sense for there to be heterogeneity in migration decisions. For example, younger and more educated people might be more likely to migrate between states \citep{17}. Families might be less likely to migrate between states \citep{16}. 

%It is important to note, however, that none of these groups are monoliths, which can have significant implications for their migration outcomes \citep{20}. 

Individuals in same-sex relationships are another group with specific preferences. Specifically, they have a preference for states with legal same-sex marriage. Using an event study empirical framework, \citet{1} found that legal same-sex marriage was a pull factor \textit{to} states for individuals in same-sex relationships. Using a probit model, \citet{12} found that lack of locally-legalized same-sex marriage was a push factor \textit{out} out of states for individuals in same-sex relationships. This is also in line with other results from LGBTQ+ economics: migration heterogeneity stemmed from legal discrimination against LGBTQ+ individuals, not sexuality.

%There is an endogeneity concern that other factors that impact same-sex marriage legalization also impact same-sex migration, but these can be controlled for using state and time fixed effects \citep{5}. 

After 2015, same-sex marriage became legal everywhere. Like how locally legal same-sex marriage decreased heterogeneity within states, federal legalization might have led to decreased heterogeneity across states for individuals in same-sex relationships. Unlike the topics discussed above, this has not been studied as extensively, and what this paper will address.




\section{Model}
\subsection{Theoretical Framework}
Following \citet{1, 12, 18}, I predict that an individual migrates between states when the utility of migrating is greater than the utility of not migrating. Under this framework, the utility of migrating depends on the utility of an individual's current state, the utility of a possible new state, and moving costs. The utilities of a given place depend on the characteristics of individual $i$ and state $s$. This can be seen in the following equation: 

\begin{equation}
U(i, g, G) = P_1(i, G) - P_2(i, g) - C(i, g, G)
\end{equation}
where $U(i, g, G)$ is the utility of migrating from state $g$ to state $G$ for person $i$, $P_1(i, G)$ is the utility of a possible new state, $P_2(i, g)$ is the utility of individual $i$'s current state, and $C(i, g, G)$ are individual $i$'s moving costs from state $g$ to $G$. When deciding whether to migrate, individual $i$ can be understood as maximizing $U(i, g, G)$:

\begin{equation}
\arg\max_{G \in S} U(i, g, G)
\end{equation}
where S is the set of all states in the United States. If $U(i, g, G)$ is maximized when $G = g$, individual $i$ chooses not to migrate.

\hfill
\break
Between 2003 and 2015, legalized same-sex marriage affected created utility differentials across states for individuals in same-sex relationships \citep{1, 12}. Legalized same-sex marriage is a benefit for individuals in same-sex relationships: it provides legal rights and social acceptance not available without it. When states legalized same-sex marriage, more individuals in same-sex relationships were pulled in and less pushed out. 

I predict federal legalization will have a different effect. By increasing geographic access to same-sex marriage, federal legalization dilutes the benefit of locally legalized same-sex marriage in a given state. When some states have same-sex marriage legalization and others don't, I would expect individuals who care about same-sex marriage legalization to get extra utility out of those states relative to other states. When all states have same-sex marriage legalization, I would expect those locally legalized states to \textit{lose} that extra utility as same-sex marriage legalization no longer differentiates them. This means that, after 2015, I would expect \textit{less} individuals in same-sex relationships to move to states that locally legalized same-sex marriage and \textit{more} to move out.

In this paper, I attempt to measure this dilution effect by studying how same-sex migration patterns changed after federal legalization in 2015. It is also important to note that I would also expect some dilution to occur every time a new state legalized same-sex marriage between 2003 and 2015. Figures \ref{fig: MA_post_trends} and \ref{fig: MA_ante_trends} in the appendix illustrate how subsequent waves of legalization impacted same-sex migration to Massachusetts, the first state that legalized same-sex marriage.

\subsection{Empirical Framework}

To test this theory, I use a triple differences specification \citep{23, 24, 25}. Triple difference models assume parallel differences and a plausible causal mechanism. In this model, I assume that same-sex marriage legalization does not impact individuals in opposite-sex relationships and exploit that federal legalization impacted all states at the same time. To understand how federal legalization might behave differently as a pull factor to a state or push factor out of a state, or have heterogeneous effects across functions $P_1$ and $P_2$ above, I run two parallel specifications. They are as follows:

\hfill
\break
%<pull eqn>
Pull Factor Regression:
\begin{equation}
\text{m}_{itg} = \gamma_t + \gamma_g + \gamma_s + \gamma_{tg} + \gamma_{ts} + \gamma_{sg} + \beta \cdot (\text{samesex}_i \times \text{post2015}_t \times \text{locally legalized}_g) 
+ X_{it} + \epsilon_{itg}
\end{equation}

\hfill
\break
Push Factor Regression:
\begin{equation}
\text{m}_{ith} = \gamma_t + \gamma_h + \gamma_s + \gamma_{th} + \gamma_{ts} + \gamma_{sh} + \beta \cdot (\text{samesex}_i \times \text{post2015}_t \times \text{locally legalized}_h) 
+ X_{it} + \epsilon_{ith}
\end{equation}

In both, outcome variable $m$ is whether individual $i$ migrated between year $t-1$ and $t$. $\gamma_t$ captures year-fixed effects for year $t$ and $\gamma_s$ captures the effect of being in relationship type $s$ (same-sex or opposite-sex). In the pull factor model, $\gamma_g$ captures state-fixed effects for the state $g$ individual $i$ lives in year $t$. In the push factor model, $\gamma_h$ captures state-fixed effects for the state $h$ individual $i$ lives in year $t-1$. All other $\gamma$ terms capture interactions between these terms. 

In the pull factor triple interaction, $samesex_i$ is a dummy variable that indicates whether individual $i$ is in a same-sex relationship, $post2015_t$ is a dummy variable that indicates whether year $t$ is on or after 2015, and $locally legalized_g$ is a dummy variables that indicates whether state $g$ had legalized same-sex marriage in year $t$. The push factor triple interaction is similar; the only difference is that $locally legalized_g$ is a dummy variables that indicates whether state $h$ had legalized same-sex marriage in year $t$. It is valid to measure the local legalization status of state $h$ in year $t$ and not year $t-1$ because migration can occur in year $t-1$ or year $t$. In both $locally legalized$ terms, a state is defined as having locally legalized marriage in year $t$ if state-specific local legalization is announced in year $t$ or earlier.

 In both, $X$ is a vector of controls designed to make the parallel trends assumption more plausible. Depending on model, $X$ includes sex, race, education, age, income, parenthood, and birth state. These follow similar controls used in similar studies by \citet{1}, \citet{3}, \citet{5}, \citet{7}, and \citet{12}. In both, probability weights are used.

In both models, the coefficient of interest is $\beta$. In the pull factor regression, $\beta$ is the expected difference in migration between individuals in same-sex and opposite-sex relationships before and after 2015 \textit{to} states that locally-legalized same-sex marriage. In the push factor regression, $\beta$ is the expected difference in migration between individuals in same-sex and opposite-sex relationships before and after 2015 \textit{from} states that locally-legalized same-sex marriage. Following the dilution effect model outlined above, I expect $\beta$ to be negative in pull factor regressions and negative in push factor regressions. This would indicate \textit{less} migration to and \textit{more} migration out of states that locally legalized same-sex marriage before 2015 after federal legalization.

To further understand the relationship between same-sex marriage legalization and migration, I test for heterogeneous effects across migration flow types, sex, and region and also for the relative importance of statewide social acceptance in migration decisions. The code I use and figures and charts I generate, can be found here: \href{https://github.com/nj-rich/same-sex-migration}{GitHub Repository}. This excludes the raw Census data I downloaded from IPUMS.

\section{Data}

This paper primarily makes use of American Community Survey (ACS) data compiled in the Integrated Public Use Microdata Series \citep{28}. The ACS is an annual nationwide survey with observations at the individual level. I pull results from surveys between 2011 and 2019. I also use data from \citet{27} for when different states announced legal same-sex marriage and from \citet{29} to measure the level of popular support of individuals in same-sex relationships in each state. Massachusetts was the first state to legalize same-sex marriage in 2003; by 2014, 37 states had legal same-sex marriage.

\begin{longtable}{|c|c|}
\hline
\textbf{State} & \textbf{Year of Same-Sex Marriage Legalization} \\
\hline
Massachusetts & 2003 \\
Connecticut & 2008 \\
Iowa & 2009 \\
New Hampshire & 2009 \\
District of Columbia & 2009 \\
Vermont & 2009 \\
New York & 2011 \\
Maine & 2012 \\
Maryland & 2012 \\
Washington & 2012 \\
California & 2013 \\
Delaware & 2013 \\
Hawaii & 2013 \\
Illinois & 2013 \\
Minnesota & 2013 \\
New Jersey & 2013 \\
New Mexico & 2013 \\
Rhode Island & 2013 \\
Alaska & 2014 \\
Arizona & 2014 \\
Colorado & 2014 \\
Idaho & 2014 \\
Indiana & 2014 \\
Kansas & 2014 \\
Montana & 2014 \\
Nevada & 2014 \\
North Carolina & 2014 \\
Oklahoma & 2014 \\
Oregon & 2014 \\
Pennsylvania & 2014 \\
South Carolina & 2014 \\
Utah & 2014 \\
Virginia & 2014 \\
West Virginia & 2014 \\
Wisconsin & 2014 \\
Wyoming & 2014 \\
Alabama & 2015 \\
Arkansas & 2015 \\
Florida & 2015 \\
Georgia & 2015 \\
Kentucky & 2015 \\
Louisiana & 2015 \\
Michigan & 2015 \\
Mississippi & 2015 \\
Missouri & 2015 \\
Nebraska & 2015 \\
North Dakota & 2015 \\
Ohio & 2015 \\
South Dakota & 2015 \\
Tennessee & 2015 \\
Texas & 2015 \\
\hline
\multicolumn{2}{p{0.8\linewidth}}{\small \textbf{Note:} This table shows the year in which same-sex marriage was \textit{most recently} legalized in each state as recorded by \citet{27}.} \\ 
\end{longtable}



The ACS dataset allows for the indirect identification of an individual’s relationship type and migration status. The \citet{27} dataset allows for the identification of a state’s same-sex marriage legalization status. The \citet{29} dataset allows for the identification of statewide popular support for individuals in same-sex relationships in 2016.\footnote{The American National Election Studies (ANES) survey is a weighted nationwide survey taken every four years with observations at the individual level. I identify statewide support for individuals in same-sex relationships by using the question: How would you rate: Gay men and lesbians? in the 2016 survey wave. Answers closer to 100 indicate higher ratings and answers closer to 0 indicate lower ratings. In this paper, I reverse this for easier interpretation.}

The ACS records if an individual is in a relationship with someone that lives in their household, and if they are, whether their partner has their same sex or not. This enables the identification of individuals in same-sex and opposite-sex relationships. Because the ACS does not explicitly identify respondents' sexuality, I use individuals in same-sex relationships to proxy LGBTQ+ individuals. It is important to note this measure is further flawed: individuals in same-sex or opposite-sex relationships with partners that live in different households will be classified as partnerless. Ultimately, about one percent of respondents identify as in same-sex relationships each year \citep{28}. The ACS also records an individual’s current state of residence and which state they lived in the year prior. This allows for the direct identification of interstate migrants. In both the push and pull factor model, I attribute a state’s legalization status to the current year, not the prior year. As stated above, I justify this by noting migration can occur in the current or prior year. Locally legalized states are those in which same-sex marriage was legalized in the given year or earlier by state law, statewide referendum, or state court. Not locally legalized states are those in which same-sex marriage has not been legalized in the given year or earlier by state law, statewide referendum, or state court. After 2015, these are the states that Obergefell v Hodges affected. About 63 percent percent of respondents lived in locally-legalized states in 2014. 

\FloatBarrier
\begin{table}

\caption{Summary Statistics for Overall Population}
\centering
\begin{tabular}[t]{rrrr}
\toprule
Year & \%/100 Female & \%/100 Same-Sex & \%/100 in Locally Legalized States\\
\midrule
2011 & 0.495 & 0.010 & 0.104\\
2012 & 0.495 & 0.011 & 0.150\\
2013 & 0.495 & 0.013 & 0.336\\
2014 & 0.496 & 0.014 & 0.628\\
2015 & 0.496 & 0.015 & 0.627\\
\addlinespace
2016 & 0.497 & 0.015 & 0.627\\
2017 & 0.497 & 0.016 & 0.627\\
2018 & 0.497 & 0.017 & 0.629\\
2019 & 0.498 & 0.017 & 0.627\\
\bottomrule
\end{tabular}
\end{table}



The ACS includes data on individuals who are not in relationships, migrated to the United States, and were born outside of the United States. Following \citet{1}, I exclude individuals not in relationships to directly compare individuals in relationships. I exclude migration to the US because I only want to compare migrants from places where the Supreme Court can affect same-sex marriage legalization. I exclude individuals born outside of the United States because I include birth state as a control. Following \citet{12}, I assume this control affects $m$. As $m$ only identifies migration between states, this control will not behave as expected for individuals born outside of the US. 

I also apply several restrictions for technical reasons. I exclude observations in which the ACS has identified errors in coding an individual’s sex or has identified an individual’s sex as missing. While researchers have noted that ACS more reliably identifies individuals in same-sex relationships after 2012, other researchers have noted that before this there were sex-coding reliability issues before this \citep{3, 5, 7, 12}. Dropping these observations ensures more reliable identification of individuals in same-sex relationships. I also drop observations with missing age, education, and race information and unknown employment status and income status. I do this because these variables are included as controls in my model. Summary statistics for these key variables can be found in table \ref{var_table_1} in this section and \ref{var_table_2} in the appendix.

\begin{table}

\caption{Summary Statistics for Main Variables (1)}
\centering
\begin{tabular}[t]{rrrrrrr}
\toprule
\multicolumn{1}{c}{ } & \multicolumn{2}{c}{\% Migrate} & \multicolumn{2}{c}{\% Female} & \multicolumn{2}{c}{\% Has Child} \\
\cmidrule(l{3pt}r{3pt}){2-3} \cmidrule(l{3pt}r{3pt}){4-5} \cmidrule(l{3pt}r{3pt}){6-7}
Year & Same-Sex & Opposite-Sex & Same-Sex & Opposite-Sex & Same-Sex & Opposite-Sex\\
\midrule
2011 & 16.247 & 10.200 & 53.615 & 49.466 & 21.219 & 48.967\\
2012 & 16.541 & 10.336 & 52.696 & 49.465 & 23.042 & 48.648\\
2013 & 17.153 & 10.690 & 52.769 & 49.433 & 21.130 & 48.583\\
2014 & 16.564 & 10.794 & 52.870 & 49.507 & 21.324 & 48.303\\
2015 & 17.084 & 10.798 & 53.401 & 49.516 & 22.397 & 47.795\\
\addlinespace
2016 & 17.548 & 10.753 & 52.082 & 49.623 & 20.692 & 47.617\\
2017 & 16.596 & 10.722 & 52.921 & 49.623 & 21.235 & 47.260\\
2018 & 18.100 & 10.601 & 52.945 & 49.630 & 20.297 & 46.952\\
2019 & 18.289 & 10.386 & 53.845 & 49.707 & 19.642 & 46.490\\
\bottomrule
\end{tabular}
\end{table}


\clearpage
\section{Results}
In the following section, I discuss the results of my analysis. For all analyses, I roughly expect “pull factor” models to follow the trends in figure \ref{fig: ex_post_diffs} and “push factor” models to follow the trends in figure \ref{fig: ex_ante_diffs}. These illustrate how I expect less individuals in same-sex relationships to move to states that “locally-legalized” same-sex marriage after 2015 and more to move out after 2015 compared to individuals in opposite-sex relationships.

I first discuss the results of the main model described in section 4.2. I then discuss migration flow heterogeneity, sex heterogeneity, and region heterogeneity. I conclude by discussing the relative importance of marriage legalization relative to social acceptance in same-sex migration decisions.

\begin{figure}[h]
    \includegraphics[width=0.75\linewidth]{outputs/summary_stats/ex_post_diffs.png}
    \centering
    \caption{}
    \label{fig: ex_post_diffs}
\end{figure}

\begin{figure}[h]
    \centering
    \includegraphics[width=0.75\linewidth]{outputs/summary_stats/ex_ante_diffs.png}
    \caption{}
    \label{fig: ex_ante_diffs}
\end{figure}

\clearpage
\subsection{Main Model} %ok still need to fix centering
In table \ref{tab: expost_model} I present main regression results from “pull factor” models and in table \ref{tab: exante_model} I present main regression results from “push factor” models. In both tables, column 1 reports the regression coefficient of a model with only state and year-fixed effects; column 2 reports the regression coefficient of a model with these fixed effects and controls for sex, race, education level, the presence of children in the household, income level, and age; and column 3 reports the regression coefficients of a model with these fixed effects, controls, and controls for an individual’s birth state. Figure \ref{fig: post_diffs} illustrates the underlying trends of the “pull factor” model while figure \ref{fig: ante_diffs} illustrates the underlying trends of the “push factor” model. 

Triple difference models require the assumption that there is an identifiable causal mechanism and parallel trends \citep{24, 25}. In earlier sections, I have described plausible causal mechanisms. However, it can be seen in figure \ref{fig: post_diffs} and figure \ref{fig: ante_diffs} that the parallel trend assumption fails. There are a number of reasons why this might have occurred. In the period between 2011 and 2015, there are significant changes in which states have “locally-legalized” same-sex marriage and which states do not. This means that one migrant identified as moving to (from) a “not-locally-legalized” state one year might have been identified as moving to (from) a “locally-legalized” state only one or two years later on. This could contribute to a violation of parallel trends as other factors, not related to same-sex marriage legalization but related to an individual's relationship type (same-sex or opposite-sex), contribute to migration pattern differences across states. Further, the percentage of individuals migrating between states in any given year is relatively low, at less than 20 percent. More noise is expected with smaller sample sizes. 


\begin{figure}[h]
    \centering
    \includegraphics[width=0.75\linewidth]{outputs/summary_stats/post_diffs.png}
    \caption{}
    \label{fig: post_diffs}
\end{figure}

\begin{figure}[h]
    \centering
    \includegraphics[width=0.75\linewidth]{outputs/summary_stats/ante_diffs.png}
    \caption{}
    \label{fig: ante_diffs}
\end{figure}

In the “pull factor” model, the coefficient of interest is negative and statistically insignificant across all three models. Further, the standard error in models 2 and 3 suggests the true population mean could be 0 or positive. The sign of the coefficients of interest is what I would expect. However this means there is close to no difference after 2015 in migration patterns between individuals in same-sex and opposite-sex relationships across locally-legalized and not-locally-legalized states. In other words, federal legalization led to minimal or no “pull factor” “dilution effect.” 

In the “push factor” model, the coefficient of interest is negative and statistically insignificant across all three models. Further, the standard error in all models suggests the true population mean could be 0 or positive. While the sign of the coefficients of interest is not what I would expect, this means there is close to no difference between the fraction of same-sex individuals who moved from states that “locally-legalized” same-sex marriage and those that did not before and after 2015 relative to opposite-sex individuals. In other words, federal legalization led to no “push factor” “dilution effect,” or just slightly to the opposite.  

Even as both main “pull factor” and “push factor”  regressions did not support my “dilution effect” hypothesis, they did so in heterogenous ways. This is puzzling: one model predicted a slight decrease in migration to states that “locally-legalized” same-sex marriage after 2015; the other predicted a slight decrease in migration from states that “locally-legalized” same-sex marriage after 2015. This would suggest same-sex marriage was less of a “pull factor” but more of a “push factor” after 2015. One plausible explanation for this is related to sample size: the pool of migrants to a given state is larger than the pool of migrants from a given state. Using similar logic as above, it is less surprising that there is more deviation for the regressions run on smaller sample sizes. Other plausible explanations include the existence of state and individual heterogeneity that obscures the effect of “local-legalization” on migration decisions, and the existence of other, more salient factors, impacting the migration decisions of individuals in same-sex relationships. This heterogeneity and potential impact of other factors is explored in the following subsections. 

\begin{table}[h]
    \centering
    \caption{Main Pull Factor Model}
    \label{tab: expost_model}
    \begin{tabular}{lc}
\multicolumn{2}{c}{Ex-Post Model} \\ \hline
 & (1) \\
VARIABLES & migrant \\ \hline
 &  \\
Constant & 3.757*** \\
 & (0.123) \\
 &  \\
Observations & 956,236,912 \\
 R-squared & 0.912 \\ \hline
\multicolumn{2}{c}{ Robust standard errors in parentheses} \\
\multicolumn{2}{c}{ *** p$<$0.01, ** p$<$0.05, * p$<$0.1} \\
\multicolumn{2}{c}{ Model 1 includes interaction terms and fixed effects only.\ Model 2 includes interaction terms, fixed effects, and controls for sex, race, education, age and income.} \\
\multicolumn{2}{c}{ )} \\
\multicolumn{2}{c}{ )} \\
\multicolumn{2}{c}{ )} \\
\end{tabular}

\end{table}
\begin{table}[h]
    \centering
    \caption{Main Push Factor Model}
    \label{tab: exante_model}
    \begin{tabular}{lccc}
\multicolumn{4}{c}{Ex-Ante Model} \\ \hline
 & (1) & (2) & (3) \\
VARIABLES & Model 1 & Model 2 & Model 3 \\ \hline
 &  &  &  \\
ante\_treatment & -0.007 & -0.004 & -0.003 \\
 & (0.010) & (0.023) & (0.025) \\
Constant & 0.100*** & 4.141*** & 4.087*** \\
 & (0.000) & (0.084) & (0.108) \\
 &  &  &  \\
Observations & 19,755 & 19,755 & 19,755 \\
 R-squared & 0.004 & 0.919 & 0.921 \\ \hline
\multicolumn{4}{c}{ Robust standard errors in parentheses} \\
\multicolumn{4}{c}{ *** p$<$0.01, ** p$<$0.05, * p$<$0.1} \\
\multicolumn{4}{c}{ See below.} \\
\end{tabular}

\end{table}

\clearpage
\subsection{Flow Heterogeneity}

One source of heterogeneity is from which type of state an individual might move in the “pull factor” model and to which type of state an individual might move in the “push factor” model. If fewer individuals in same-sex relationships move \textit{between} “locally-legalized” states after 2015, that could make the “push factor” model coefficient negative even if more individuals in same-sex relationships move from “locally-legalized” to “not-locally-legalized” states after 2015. This could help explain the opposite-than-expected main “push factor” regression results.

Tables \ref{tab: ffed_expost_model}, \ref{tab: flocal_expost_model}, and \ref{tab: fstay_expost_model} present regression results from “pull factor” models in which the outcome variable indicates whether an individual migrated from a state with “locally-legalized” same-sex marriage, migrated from a state that had “not-locally-legalized” same-sex marriage, or whether an individual did not migrate from a different state. Tables \ref{tab: tfed_exante_model}, \ref{tab: tlocal_exante_model}, and \ref{tab: tstay_exante_model} present regression results from “push factor” models in which the outcome variable indicates whether an individual migrated to a state with “locally-legalized” same-sex marriage, migrated to a state that had “not-locally-legalized” same-sex marriage, or whether an individual did not migrate to a different state. All tables except tables \ref{tab: tfed_exante_model} and \ref{tab: tlocal_exante_model} can be found in the appendix. Figure \ref{fig: flows_post_diffs} illustrates the underlying trends of the “pull factor” model while figure \ref{fig: flows_ante_diffs} illustrates the underlying trends of the “push factor” model with the modified outcome variables.

As these regressions are still triple difference models, assumptions of a plausible causal mechanism and parallel trends are still needed \citep{24, 25}. In line with the “dilution effect”, I would expect there to be slightly less migration between states that “locally legalized” same-sex marriage after 2015. I expect this as previously “not-locally-legalized” states become marginally more attractive for those in same-sex relationships. On the other hand, I would expect there to be more migration from states that “locally-legalized” same-sex marriage to states that had “not-locally-legalized” same-sex marriage after 2015, and less migration from states that had “not-locally-legalized” same-sex marriage to states that “locally-legalized” same-sex marriage after 2015. Figures \ref{fig: flows_post_diffs} and \ref{fig: flows_ante_diffs} illustrate that parallel trends do not hold for any of the modified outcome variables. However, it is important to note that the to (from) “not-locally-legalized” trends appear significantly less noisy.

\begin{figure}[h]
    \centering
    \includegraphics[width=1\linewidth]{outputs/summary_stats/flows_post_diffs.png}
    \caption{}
    \label{fig: flows_post_diffs}
\end{figure}
\begin{figure}[h]
    \centering
    \includegraphics[width=1\linewidth]{outputs/summary_stats/flows_ante_diffs.png}
    \caption{}
    \label{fig: flows_ante_diffs}
\end{figure}

 The two coefficients of interest in this exercise are those in column 3  of table \ref{tab: tfed_exante_model} and table \ref{tab: tlocal_exante_model}. The coefficient in column 3 of table \ref{tab: tfed_exante_model} is negative and statistically insignificant and the coefficient in column 3 of table \ref{tab: tlocal_exante_model} is positive and statistically insignificant. The standard error in both columns suggests the true population mean could be 0 or of the opposite sign. This indicates that those in same-sex relationships were slightly less likely to move from “locally-legalized” states to “not-locally-legalized” states and slightly more likely to move to other “locally-legalized” states after 2015. This is the opposite of what I would have predicted under my theory. This then suggests that flow heterogeneity, as measured, does not help explain away the opposite-than-expected results in the main “push factor” regression. 

\begin{table}[p] %finagling to get formatting right
    \centering
    \caption{Push Factor Model: To Not Locally Legalized}
    \label{tab: tfed_exante_model}
    \begin{tabular}{lccc}
\hline
 & (1) & (2) & (3) \\
VARIABLES & Model 1 & Model 2 & Model 3 \\ \hline
 &  &  &  \\
$\hat{\beta_2}$ & 0.031*** & 0.025 & 0.015 \\
 & (0.009) & (0.028) & (0.031) \\
Constant & 0.100*** & 2.094*** & 1.977*** \\
 & (0.000) & (0.314) & (0.357) \\
 &  &  &  \\
Observations & 19,755 & 19,755 & 19,755 \\
 R-squared & 0.045 & 0.519 & 0.551 \\ \hline
\multicolumn{4}{c}{ Robust standard errors in parentheses} \\
\multicolumn{4}{c}{ *** p$<$0.01, ** p$<$0.05, * p$<$0.1} \\
\multicolumn{4}{p{0.6\linewidth}}{\footnotesize Column 1 reports the regression coefficient of a model with state, year, and relationship-type fixed effects including corresponding interactions; column 2 reports the regression coefficient of a model with these fixed effects and controls for sex, race, education level, the presence of children in the household, income level, and age; and column 3 reports the regression coefficients of a model with these fixed effects, controls, and controls for an individual’s birth state.} \\
\end{tabular}

\end{table}

\begin{table}[p]
    \centering
    \caption{Push Factor Model: To Locally Legalized}
    \label{tab: tlocal_exante_model}
    \begin{tabular}{lccc}
\multicolumn{4}{c}{To Locally Legalized} \\ \hline
 & (1) & (2) & (3) \\
VARIABLES & Model 1 & Model 2 & Model 3 \\ \hline
 &  &  &  \\
ante\_treatment & 0.024** & 0.020 & 0.012 \\
 & (0.010) & (0.022) & (0.030) \\
Constant & 0.001*** & 2.048*** & 2.110*** \\
 & (0.000) & (0.348) & (0.368) \\
 &  &  &  \\
Observations & 19,755 & 19,755 & 19,755 \\
 R-squared & 0.048 & 0.516 & 0.541 \\ \hline
\multicolumn{4}{c}{ Robust standard errors in parentheses} \\
\multicolumn{4}{c}{ *** p$<$0.01, ** p$<$0.05, * p$<$0.1} \\
\end{tabular}

\end{table}

%\subsection{State Comparisons- ie something like CA v TX}

\clearpage
\subsection{Sex Heterogeneity}

Another source of heterogeneity is an individual’s sex. Research by \citet{1} and \citet{12} indicates same-sex marriage legalization could lead to larger behavioral changes for men than women in same-sex relationships. If federal legalization impacts men as expected but women differently, that could impact the main regression results.

Tables \ref{tab: male_expost_model} and \ref{tab: female_expost_model} present regression results from “pull factor” models split by sex. Table \ref{tab: male_exante_model} and \ref{tab: female_expost_model} present regression results from “push factor” models split by sex. Figure \ref{fig: sex_post_diffs} illustrates the underlying trends of the “pull factor” model while figure \ref{fig: sex_ante_diffs} illustrates the underlying trends of the “push factor” model by sex.

As these regressions are still triple difference models, assumptions of a plausible causal mechanism and parallel trends are still needed \citep{24, 25}. Like in the main model, plausible causal mechanisms are explained in earlier sections. Figures \ref{fig: sex_post_diffs} and \ref{fig: sex_ante_diffs} illustrate that parallel trends hold for men but not women. Further, the “pull factor” figure for men roughly corresponds with what I predicted.

\begin{figure}[p]
    \centering
    \includegraphics[width=1\linewidth]{outputs/summary_stats/sex_post_diffs.png}
    \caption{}
    \label{fig: sex_post_diffs}
\end{figure}
\begin{figure}[p]
    \centering
    \includegraphics[width=1\linewidth]{outputs/summary_stats/sex_ante_diffs.png}
    \caption{}
    \label{fig: sex_ante_diffs}
\end{figure}

The two coefficients of interest in this exercise are those in column 3 of the push factor models. While it is interesting that the coefficients for men and women have opposite signs, even as they are all statistically insignificant, this does not provide greater insight why the main “push” model coefficients are negative: the model for men follows the main model coefficients. This then suggests that sex heterogeneity, as measured, does not help explain away the opposite-than-expected results in the main “push factor” regression.

\clearpage %finagling to get formatting right
\begin{table}[p] %maybe put some of these in the appendix
    \centering
    \caption{Pull Factor Model: Male}
    \label{tab: male_expost_model}
    \begin{tabular}{lccc}
\multicolumn{4}{c}{Ex-Post Model: Male} \\ \hline
 & (1) & (2) & (3) \\
VARIABLES & Model 1 & Model 2 & Model 3 \\ \hline
 &  &  &  \\
post\_treatment & 0.014 & 0.007 & 0.008 \\
 & (0.009) & (0.038) & (0.043) \\
Constant & 0.102*** & 3.971*** & 3.826*** \\
 & (0.000) & (0.125) & (0.102) \\
 &  &  &  \\
Observations & 17,816 & 17,816 & 17,816 \\
 R-squared & 0.004 & 0.912 & 0.917 \\ \hline
\multicolumn{4}{c}{ Robust standard errors in parentheses} \\
\multicolumn{4}{c}{ *** p$<$0.01, ** p$<$0.05, * p$<$0.1} \\
\multicolumn{4}{c}{ See below.} \\
\end{tabular}

\end{table}
\begin{table}[p]
    \centering
    \caption{Pull Factor Model: Female}
    \label{tab: female_expost_model}
    \begin{tabular}{lccc}
\multicolumn{4}{c}{Ex-Post Model: Female} \\ \hline
 & (1) & (2) & (3) \\
VARIABLES & Model 1 & Model 2 & Model 3 \\ \hline
 &  &  &  \\
post\_treatment & 0.006 & -0.007 & -0.008 \\
 & (0.012) & (0.030) & (0.033) \\
Constant & 0.104*** & 3.856*** & 3.753*** \\
 & (0.000) & (0.102) & (0.098) \\
 &  &  &  \\
Observations & 17,942 & 17,942 & 17,942 \\
 R-squared & 0.005 & 0.916 & 0.920 \\ \hline
\multicolumn{4}{c}{ Robust standard errors in parentheses} \\
\multicolumn{4}{c}{ *** p$<$0.01, ** p$<$0.05, * p$<$0.1} \\
\end{tabular}

\end{table}
\begin{table}[p] %maybe put on top
    \centering
    \caption{Push Factor Model: Male}
    \label{tab: male_exante_model}
    \begin{tabular}{lccc}
\multicolumn{4}{c}{Ex-Ante Model: Male} \\ \hline
 & (1) & (2) & (3) \\
VARIABLES & Model 1 & Model 2 & Model 3: Male \\ \hline
 &  &  &  \\
ante\_treatment & -0.009 & -0.005 & -0.004 \\
 & (0.010) & (0.035) & (0.039) \\
Constant & 0.099*** & 4.002*** & 3.978*** \\
 & (0.000) & (0.089) & (0.100) \\
 &  &  &  \\
Observations & 17,816 & 17,816 & 17,816 \\
 R-squared & 0.004 & 0.908 & 0.909 \\ \hline
\multicolumn{4}{c}{ Robust standard errors in parentheses} \\
\multicolumn{4}{c}{ *** p$<$0.01, ** p$<$0.05, * p$<$0.1} \\
\multicolumn{4}{c}{ See below.} \\
\end{tabular}

\end{table}
\begin{table}[p] %own dedicated page
    \centering
    \caption{Push Factor Model: Female}
    \label{tab: female_exante_model}
    \begin{tabular}{lccc}
\multicolumn{4}{c}{Ex-Post Model: Female} \\ \hline
 & (1) & (2) & (3) \\
VARIABLES & Model 1 & Model 2 & Model 3 \\ \hline
 &  &  &  \\
ante\_treatment & -0.005 & 0.001 & 0.002 \\
 & (0.012) & (0.030) & (0.037) \\
Constant & 0.102*** & 3.914*** & 3.878*** \\
 & (0.000) & (0.071) & (0.092) \\
 &  &  &  \\
Observations & 17,942 & 17,942 & 17,942 \\
 R-squared & 0.004 & 0.913 & 0.914 \\ \hline
\multicolumn{4}{c}{ Robust standard errors in parentheses} \\
\multicolumn{4}{c}{ *** p$<$0.01, ** p$<$0.05, * p$<$0.1} \\
\end{tabular}

\end{table}

\clearpage
\subsection{Region Heterogeneity}

Another source of heterogeneity is from which region an individual comes from in the “push factor” model and to which region an individual is in the “pull factor” model. While the main model includes state-level fixed effects, it does not directly control for distance between states. \citep{1} suggests distance impacts what type of state an individual moves to (from), as individuals are less likely to move to (from) states further away. If an individual’s region has relatively more “locally legalized” states, that could make the  “push factor” coefficient less positive as “not locally legalized” states are relatively further away. If an individual’s region has both “locally legalized” and “not locally legalized” states, coefficients of interest should be as predicted in line with the “dilution effect.” This heterogeneity could help explain the opposite-than-expected main “push factor” regression results. Summary statistics for region can be found in table \ref{region_1} in this section and table \ref{region_2} in the appendix. Notably, a lower fraction of individuals in same-sex relationships live in the midwest and south, where less states legalized same-sex marriage earlier.

\begin{table}

\caption{Summary Statistics by Region (1)}
\centering
\begin{tabular}[t]{rrrrr}
\toprule
\multicolumn{1}{c}{ } & \multicolumn{2}{c}{\% in the Midwest} & \multicolumn{2}{c}{\% in the South} \\
\cmidrule(l{3pt}r{3pt}){2-3} \cmidrule(l{3pt}r{3pt}){4-5}
Year & Same-Sex & Opposite-Sex & Same-Sex & Opposite-Sex\\
\midrule
2011 & 19.904 & 25.041 & 33.475 & 37.691\\
2012 & 19.891 & 25.036 & 33.792 & 37.746\\
2013 & 18.736 & 25.041 & 34.482 & 37.755\\
2014 & 20.224 & 24.922 & 34.505 & 37.845\\
2015 & 18.593 & 24.829 & 34.929 & 37.975\\
\addlinespace
2016 & 18.781 & 24.794 & 35.556 & 38.011\\
2017 & 19.101 & 24.664 & 35.537 & 38.047\\
2018 & 19.323 & 24.627 & 35.796 & 37.955\\
2019 & 18.446 & 24.437 & 36.848 & 38.101\\
\bottomrule
\end{tabular}
\end{table}


Tables \ref{tab: midwest_expost_model}, \ref{tab: south_expost_model}, \ref{tab: west_expost_model}, and \ref{tab: northeast_expost_model} presents regression results from “pull factor” models split by region. Tables \ref{tab: midwest_exante_model}, \ref{tab: south_exante_model}, \ref{tab: west_exante_model}, and \ref{tab: northeast_exante_model}  presents regression results from “push factor” models split by region. All tables except \ref{tab: midwest_exante_model} and \ref{tab: south_exante_model} can be found in the appendix. Figure \ref{fig: region_post_diffs} illustrates the underlying trends of the “pull factor” model while figure \ref{fig: region_ante_diffs} illustrates the underlying trends of the “push factor” model by region. The northeast and west “pull factor” figures are cut off after 2013 as all states in the northeast and west locally legalized same-sex marriage by 2014.

As these regressions are still triple difference models, assumptions of a plausible causal mechanism and parallel trends are still needed \citep{24, 25}. Like in the main model, plausible causal mechanisms are explained in earlier sections, albeit with further elaboration in the first paragraph of this subsection. Figures \ref{fig: region_post_diffs} and \ref{fig: region_ante_diffs} illustrate that parallel trends roughly hold for the midwest and south “pull factor” models and for all “push factor” models. 

\begin{figure}[p]
    \centering
    \includegraphics[width=1\linewidth]{outputs/summary_stats/region_post_diffs.png}
    \caption{}
    \label{fig: region_post_diffs}
\end{figure}

\begin{figure}[p]
    \centering
    \includegraphics[width=1\linewidth]{outputs/summary_stats/region_ante_diffs.png}
    \caption{}
    \label{fig: region_ante_diffs}
\end{figure}

The coefficients of interest are in columns 3 of the midwest and south “push factor” model. This is because only the south and midwest had states that never “locally legalized” same-sex marriage. Both coefficients are negative and statistically insignificant. This then suggests that region heterogeneity, as measured, also does not help explain away to the opposite-than-expected results in the main “push factor” regression. 

\clearpage
\begin{table}[p] %finagling to get formatting to work
    \centering
    \caption{Push Factor Model: Midwest}
    \label{tab: midwest_exante_model}
    \begin{tabular}{lccc}
\multicolumn{4}{c}{Ex-Ante Model: Midwest} \\ \hline
 & (1) & (2) & (3) \\
VARIABLES & Model 1 & Model 2 & Model 3 \\ \hline
 &  &  &  \\
ante\_treatment & -0.003 & 0.017 & -0.004 \\
 & (0.017) & (0.044) & (0.048) \\
Constant & 0.095*** & 4.167*** & 4.651*** \\
 & (0.000) & (0.159) & (0.366) \\
 &  &  &  \\
Observations & 4,684 & 4,684 & 4,684 \\
 R-squared & 0.002 & 0.917 & 0.924 \\ \hline
\multicolumn{4}{c}{ Robust standard errors in parentheses} \\
\multicolumn{4}{c}{ *** p$<$0.01, ** p$<$0.05, * p$<$0.1} \\
\end{tabular}

\end{table}
\begin{table}[p]
    \centering
    \caption{Push Factor Model: South}
    \label{tab: south_exante_model}
    \begin{tabular}{lccc}
\multicolumn{4}{c}{Ex-Ante Model: South} \\ \hline
 & (1) & (2) & (3) \\
VARIABLES & Model 1 & Model 2 & Model 3 \\ \hline
 &  &  &  \\
ante\_treatment & -0.003 & -0.061 & -0.057 \\
 & (0.017) & (0.045) & (0.048) \\
Constant & 0.100*** & 4.092*** & 4.012*** \\
 & (0.000) & (0.128) & (0.156) \\
 &  &  &  \\
Observations & 6,807 & 6,807 & 6,807 \\
 R-squared & 0.002 & 0.934 & 0.937 \\ \hline
\multicolumn{4}{c}{ Robust standard errors in parentheses} \\
\multicolumn{4}{c}{ *** p$<$0.01, ** p$<$0.05, * p$<$0.1} \\
\multicolumn{4}{c}{ See below.} \\
\end{tabular}

\end{table}



\clearpage
\subsection{Social Acceptance}

Above, I have attempted to use heterogeneity to explain why some of my results were opposite to what was expected. I have not attempted to explain why my results are insignificant. One argument for why they are insignificant is that there are other, more important, factors impacting the migration decisions of individuals in same-sex relationships. One of these factors could be statewide support for individuals in same-sex relationships. If individuals in same-sex relationships index on perceived social acceptance more than legal same-sex marriage, social acceptance could dominate marriage-legalization effects.

Table \ref{tab: pop_support} displays state-wide support for individuals in same-sex relationships in 2016. Larger values indicate less support while smaller values indicate more support. Data comes from \citet{29}.

\begin{spacing}{1}
\begin{longtable}{|c|c|}
\caption{Support for Gay Men and Lesbian Women by State in 2016}
\label{tab: pop_support}
\hline
\textbf{State} & \textbf{Popular Support}\\
\hline
Alaska & 40\\
North Dakota & 45\\
Alabama & 46\\
Maine & 48\\
West Virginia & 48\\
Arkansas & 49\\
Oklahoma & 50\\
Virginia & 51\\
South Carolina & 51\\
Texas & 52\\
South Dakota & 52\\
Mississippi & 52\\
Ohio & 53\\
Indiana & 54\\
Tennessee & 54\\
Missouri & 55\\
North Carolina & 55\\
Georgia & 56\\
Iowa & 56\\
Nebraska & 57\\
Maryland & 58\\
Louisiana & 58\\
Michigan & 58\\
Florida & 59\\
Kentucky & 59\\
Montana & 59\\
Idaho & 60\\
Kansas & 60\\
Wisconsin & 61\\
Illinois & 61\\
Colorado & 61\\
Nevada & 62\\
Pennsylvania & 62\\
Utah & 62\\
Minnesota & 62\\
New Mexico & 62\\
Arizona & 63\\
Rhode Island & 63\\
Hawaii & 64\\
Washington & 64\\
Delaware & 64\\
California & 69\\
New Jersey & 70\\
New York & 70\\
Oregon & 70\\
Connecticut & 71\\
Wyoming & 72\\
Vermont & 72\\
Massachusetts & 76\\
New Hampshire & 77\\
District of Columbia & 83\\
\hline
\multicolumn{2}{p{0.8\linewidth}}{\small \textbf{Note:} Support levels represent a weighted average of responses to the question: "How would you rate gay men and lesbians?" Values closer to 100 indicate \textit{more} support; values closer to 0 indicate \textit{less} support.} \\ 
\end{longtable}
\end{spacing}


In the following regressions, I interact this measure for popular support with the triple difference term from my main model. I do not include popular support as a separate control as popular support should be captured by state-fixed effects. If popular support does play a larger role, I expect the “pull factor” coefficient of interest to be more negative and the “push factor” coefficient to be more positive. I expect this because I expect less popular support to result in less migration to “locally-legalized” states after 2015 and more migration out of “locally-legalized” states after 2015. 

\hfill
\break
%<pull eqn>
Social Acceptance Pull Factor Regression:
\begin{equation}
\begin{aligned}
\text{m}_{itg} &= \gamma_t + \gamma_g + \gamma_s + \gamma_{tg} + \gamma_{ts} + \gamma_{sg} \\
&\quad + \beta \cdot (\text{support}_g \times \text{samesex}_i \times \text{post2015}_t \times \text{locally legalized}_g) \\
&\quad + X_{it} + \epsilon_{itg}
\end{aligned}
\end{equation}

\hfill
\break
Social Acceptance Push Factor Regression:
\begin{equation}
\begin{aligned}
\text{m}_{ith} &= \gamma_t + \gamma_h + \gamma_s + \gamma_{th} + \gamma_{ts} + \gamma_{sh} \\
&\quad + \beta \cdot (\text{support}_h \times \text{samesex}_i \times \text{post2015}_t \times \text{locally legalized}_h) \\
&\quad + X_{it} + \epsilon_{ith}
\end{aligned}
\end{equation}

Table \ref{tab: popsupport_expost_model} reports results from the social acceptance “pull factor” model. Table \ref{tab: popsupport_exante_model} reports results from the social acceptance “push factor” model. All coefficients are larger in magnitude than in the main model; however, the coefficients in the “push factor” model are larger in the opposite direction than predicted. This could indicate that legalization and popular support capture similar phenomena, both of which result in different outcomes than expected in the “push factor” model.

\begin{table}[h] %maybe put on top
    \centering
    \caption{Pull Factor Model: Popular Support}
    \label{tab: popsupport_expost_model}
    \begin{tabular}{lccc}
\multicolumn{4}{c}{Ex-Post Model - Inverse Popular Support} \\ \hline
 & (1) & (2) & (3) \\
VARIABLES & Model 1 & Model 2 & Model 3 \\ \hline
 &  &  &  \\
post\_treatment & -0.024 & -0.032 & -0.030 \\
 & (0.017) & (0.054) & (0.068) \\
Constant & 0.103*** & 4.063*** & 3.913*** \\
 & (0.000) & (0.131) & (0.108) \\
 &  &  &  \\
Observations & 19,755 & 19,755 & 19,755 \\
 R-squared & 0.005 & 0.922 & 0.927 \\ \hline
\multicolumn{4}{c}{ Robust standard errors in parentheses} \\
\multicolumn{4}{c}{ *** p$<$0.01, ** p$<$0.05, * p$<$0.1} \\
\multicolumn{4}{c}{ See below.} \\
\end{tabular}

\end{table}
\begin{table}[h]
    \centering
    \caption{Push Factor Model: Popular Support}
    \label{tab: popsupport_exante_model}
    \begin{tabular}{lccc}
\multicolumn{4}{c}{Ex-Ante Model - Anti Popular Support} \\ \hline
 & (1) & (2) & (3) \\
VARIABLES & Model 1 & Model 2 & Model 3 \\ \hline
 &  &  &  \\
ante\_treatment & -0.011 & -0.060 & -0.057 \\
 & (0.022) & (0.048) & (0.054) \\
Constant & 0.100*** & 4.142*** & 4.087*** \\
 & (0.000) & (0.084) & (0.108) \\
 &  &  &  \\
Observations & 19,755 & 19,755 & 19,755 \\
 R-squared & 0.004 & 0.919 & 0.921 \\ \hline
\multicolumn{4}{c}{ Robust standard errors in parentheses} \\
\multicolumn{4}{c}{ *** p$<$0.01, ** p$<$0.05, * p$<$0.1} \\
\multicolumn{4}{c}{ See below.} \\
\end{tabular}

\end{table}

\section{Conclusion}

In a landmark civil rights ruling, the Supreme Court federally legalized same-sex marriage in 2015. This begs the question, how did individuals in same-sex relationships respond to the geographic expansion of their rights? In this paper, I investigated to what extent migration patterns of individuals in same-sex and opposite-sex relationships converged after 2015. I did this by comparing migration patterns to states that locally legalized same-sex marriage and those that did not. I added to the literature by addressing the effects of federal legalization after 2015 and introducing the idea of a “dilution effect.”

I found statistically insignificant differences between where individuals in same-sex and opposite-sex relationships move before and after 2015. Specifically, the main “pull model” coefficient of interest was negative and statistically insignificant and the main “push model” coefficient of interest was negative and statistically insignificant. While the sign of the main “pull model” coefficient of interest was expected, the sign of the main “push model” coefficient of interest was not. These results held across model specifications designed to account for flow heterogeneity, sex heterogeneity, region heterogeneity, and the impact of other factors such as social acceptance. 

I suspect I get these results because of poor data quality and because the relationship between same-sex marriage and migration is relatively weak. While the ACS has a large sample, the fraction of those who move between states in a given year is relatively small. While same-sex marriage legalization might have some impact on individuals’ migration decisions, it might matter more in other contexts. While the ACS has a large sample, the fraction that are identified as in same-sex relationships is relatively small, at about 1 percent. Similarly, the fraction identified as migrants is also relatively small, at less than 20 percent. This also might help explain why the “push factor” model coefficient of interest is opposite than expected: the pool of migrants to a given state is larger than the pool of migrants from a given state. Further, while same-sex marriage legalization might have some impact on individuals’ migration decisions, it might matter more in other contexts. Between 2011 and 2019, the percentage of those the ACS identified as in same-sex relationships almost doubled from 1 percent to 1.7 percent.

These null results hold important implications. There might still be identifiable relationships between federal same-sex marriage legalization and migration; the methods used here simply do not identify them. Following \citet{1, 12}, it might be worthwhile to further consider the implications of migration distance on any migration model. Following \citet{15}, it might be worthwhile to consider the implications of same-sex marriage legalization on migration levels. Following my observations, it might be worthwhile to consider the implications of federal same-sex marriage legalization on other factors, such as level of individuals identified as in same-sex relationships. Finally, it might be worthwhile to consider if the “dilution effect” holds in other contexts, or in similar contexts but using other measurement tools.

I wrote earlier that the legal landscape for individuals in same-sex relationships has changed dramatically in the United States. Even though same-sex marriage is now the law of the land, there continue to be evolving laws and norms affecting LGBT+ people. While understanding these changes in the context of discrimination, law, politics, and culture continues to be important, economic frameworks also provide insight. Further research continues to be useful.

\newpage
% References section
\bibliographystyle{chicago}
\bibliography{Drafting/thesis_bibliography}

\newpage
\appendix
\FloatBarrier
\section{More Summary Statistics}

\begin{table}

\caption{Summary Statistics by Region (2)}
\label{region_2} %added
\centering
\begin{tabular}[t]{rrrrr}
\toprule
\multicolumn{1}{c}{ } & \multicolumn{2}{c}{\% in the West} & \multicolumn{2}{c}{\% in the Northeast} \\
\cmidrule(l{3pt}r{3pt}){2-3} \cmidrule(l{3pt}r{3pt}){4-5}
Year & Same-Sex & Opposite-Sex & Same-Sex & Opposite-Sex\\
\midrule
2011 & 27.132 & 20.213 & 19.489 & 17.055\\
2012 & 26.721 & 20.257 & 19.596 & 16.961\\
2013 & 26.781 & 20.332 & 20.001 & 16.872\\
2014 & 25.497 & 20.475 & 19.774 & 16.758\\
2015 & 26.558 & 20.576 & 19.919 & 16.620\\
\addlinespace
2016 & 27.304 & 20.784 & 18.359 & 16.411\\
2017 & 26.843 & 20.822 & 18.519 & 16.467\\
2018 & 26.159 & 21.015 & 18.721 & 16.403\\
2019 & 26.867 & 21.149 & 17.840 & 16.312\\
\bottomrule
\end{tabular}
\end{table}


\begin{landscape}
\begin{table}[htbp]

\caption{Summary Statistics for Main Variables (2)}
\label{var_table_2} %added
\centering
\begin{tabular}[t]{rrrrrrrrr}
\toprule
\multicolumn{1}{c}{ } & \multicolumn{2}{c}{Mean Age} & \multicolumn{2}{c}{\% College} & \multicolumn{2}{c}{\% White} & \multicolumn{2}{c}{Mean Income} \\
\cmidrule(l{3pt}r{3pt}){2-3} \cmidrule(l{3pt}r{3pt}){4-5} \cmidrule(l{3pt}r{3pt}){6-7} \cmidrule(l{3pt}r{3pt}){8-9}
Year & Same-Sex & Different-Sex & Same-Sex & Different-Sex & Same-Sex & Different-Sex & Same-Sex & Different-Sex\\
\midrule
2011 & 47 & 50 & 99 & 99 & 86 & 88 & 50335 & 44533\\
2012 & 47 & 50 & 99 & 100 & 86 & 88 & 51139 & 45858\\
2013 & 48 & 50 & 99 & 99 & 86 & 88 & 52174 & 47460\\
2014 & 47 & 50 & 99 & 99 & 85 & 88 & 54609 & 48671\\
2015 & 47 & 51 & 99 & 99 & 85 & 88 & 54834 & 50662\\
\addlinespace
2016 & 47 & 51 & 99 & 99 & 84 & 87 & 57084 & 52100\\
2017 & 47 & 51 & 99 & 99 & 83 & 87 & 57694 & 53684\\
2018 & 47 & 51 & 99 & 99 & 83 & 87 & 58370 & 55735\\
2019 & 45 & 51 & 99 & 99 & 81 & 87 & 60205 & 58420\\
\bottomrule
\multicolumn{9}{p{0.8\linewidth}}{\footnotesize Statistics are derived from the American Community Survey compiled by \citet{28}.} \\
\end{tabular}
\end{table}

\end{landscape}


\FloatBarrier
\newpage
\section{Additional Difference Charts}
%age
\begin{figure}[h]
    \centering
    \includegraphics[width=0.75\linewidth]{outputs/summary_stats/age_post_diffs.png}
    \caption{}
    \label{}
\end{figure}

\begin{figure}[h]
    \centering
    \includegraphics[width=0.75\linewidth]{outputs/summary_stats/age_ante_diffs.png}
    \caption{}
    \label{fig: fig:enter-label}
\end{figure}


%child
\begin{figure}[h]
    \centering
    \includegraphics[width=0.75\linewidth]{outputs/summary_stats/child_post_diffs.png}
    \caption{}
    \label{fig: fig:enter-label}
\end{figure}

\begin{figure}[h]
    \centering
    \includegraphics[width=0.75\linewidth]{outputs/summary_stats/child_ante_diffs.png}
    \caption{}
    \label{fig: fig:enter-label}
\end{figure}

%Education
\begin{figure}[h]
    \centering
    \includegraphics[width=0.75\linewidth]{outputs/summary_stats/educ_post_diffs.png}
    \caption{}
    \label{fig: fig:enter-label}
\end{figure}

\begin{figure}[h]
    \centering
    \includegraphics[width=0.75\linewidth]{outputs/summary_stats/educ_ante_diffs.png}
    \caption{}
    \label{fig: fig:enter-label}
\end{figure}

%income
\begin{figure}[h]
    \centering
    \includegraphics[width=0.75\linewidth]{outputs/summary_stats/inc_post_diffs.png}
    \caption{}
    \label{fig: fig:enter-label}
\end{figure}

\begin{figure}[h]
    \centering
    \includegraphics[width=0.75\linewidth]{outputs/summary_stats/inc_ante_diffs.png}
    \caption{}
    \label{fig: fig:enter-label}
\end{figure}

%regression tables
\FloatBarrier
\newpage
\section{Additional Regression Tables}
%additional flow tables
\begin{table}[h] %maybe put on top
    \centering
    \caption{Push Factor Model: To Same State}
    \label{tab: tstay_exante_model}
    \input{outputs/regressions/tstay_exante_model}
\end{table}
\begin{table}[h]
    \centering
    \caption{Pull Factor Model: From Not Locally Legalized}
    \label{tab: ffed_expost_model}
    \begin{tabular}{lccc}
\multicolumn{4}{c}{From Not Locally Legalized} \\ \hline
 & (1) & (2) & (3) \\
VARIABLES & Model 1 & Model 2 & Model 3 \\ \hline
 &  &  &  \\
post\_treatment & 0.035*** & 0.016 & 0.005 \\
 & (0.010) & (0.029) & (0.034) \\
Constant & 0.103*** & 1.968*** & 1.812*** \\
 & (0.000) & (0.325) & (0.342) \\
 &  &  &  \\
Observations & 19,755 & 19,755 & 19,755 \\
 R-squared & 0.047 & 0.531 & 0.557 \\ \hline
\multicolumn{4}{c}{ Robust standard errors in parentheses} \\
\multicolumn{4}{c}{ *** p$<$0.01, ** p$<$0.05, * p$<$0.1} \\
\multicolumn{4}{c}{ See below.} \\
\end{tabular}

\end{table}
\begin{table}[h] %maybe put on top
    \centering
    \caption{Pull Factor Model: From Locally Legalized}
    \label{tab: flocal_expost_model}
    \begin{tabular}{lccc}
\hline
 & (1) & (2) & (3) \\
VARIABLES & Model 1 & Model 2 & Model 3 \\ \hline
 &  &  &  \\
$\hat{\beta_1}$ & -0.025** & -0.015 & -0.004 \\
 & (0.010) & (0.024) & (0.030) \\
Constant & 0.000*** & 2.095*** & 2.102*** \\
 & (0.000) & (0.350) & (0.352) \\
 &  &  &  \\
Observations & 19,755 & 19,755 & 19,755 \\
 R-squared & 0.046 & 0.511 & 0.548 \\ \hline
\multicolumn{4}{c}{ Robust standard errors in parentheses} \\
\multicolumn{4}{c}{ *** p$<$0.01, ** p$<$0.05, * p$<$0.1} \\
\multicolumn{4}{p{0.6\linewidth}}{\footnotesize Column 1 reports the regression coefficient of a model with state, year, and relationship-type fixed effects including corresponding interactions; column 2 reports the regression coefficient of a model with these fixed effects and controls for sex, race, education level, the presence of children in the household, income level, and age; and column 3 reports the regression coefficients of a model with these fixed effects, controls, and controls for an individual’s birth state.} \\
\end{tabular}

\end{table}
\begin{table}[h]
    \centering
    \caption{Pull Factor Model: From the Same State}
    \label{tab: fstay_expost_model}
    \begin{tabular}{lccc}
\multicolumn{4}{c}{From the Same State} \\ \hline
 & (1) & (2) & (3) \\
VARIABLES & Model 1 & Model 2 & Model 3 \\ \hline
 &  &  &  \\
post\_treatment & 0.010 & 0.001 & 0.002 \\
 & (0.008) & (0.026) & (0.030) \\
Constant & 0.897*** & -3.063*** & -2.913*** \\
 & (0.000) & (0.131) & (0.108) \\
 &  &  &  \\
Observations & 19,755 & 19,755 & 19,755 \\
 R-squared & 0.005 & 0.922 & 0.927 \\ \hline
\multicolumn{4}{c}{ Robust standard errors in parentheses} \\
\multicolumn{4}{c}{ *** p$<$0.01, ** p$<$0.05, * p$<$0.1} \\
\end{tabular}

\end{table}

%additional region tables
\begin{table}[h] %maybe put on top
    \centering
    \caption{Pull Factor Model: Midwest}
    \label{tab: midwest_expost_model}
    \begin{tabular}{lccc}
\multicolumn{4}{c}{Ex-Post Model: Midwest} \\ \hline
 & (1) & (2) & (3) \\
VARIABLES & Model 1 & Model 2 & Model 3 \\ \hline
 &  &  &  \\
post\_treatment & 0.011 & -0.003 & -0.003 \\
 & (0.017) & (0.062) & (0.057) \\
Constant & 0.080*** & 3.771*** & 3.405*** \\
 & (0.000) & (0.126) & (0.369) \\
 &  &  &  \\
Observations & 4,486 & 4,486 & 4,486 \\
 R-squared & 0.002 & 0.940 & 0.942 \\ \hline
\multicolumn{4}{c}{ Robust standard errors in parentheses} \\
\multicolumn{4}{c}{ *** p$<$0.01, ** p$<$0.05, * p$<$0.1} \\
\end{tabular}

\end{table}
\begin{table}[h]
    \centering
    \caption{Pull Factor Model: South}
    \label{tab: south_expost_model}
    \begin{tabular}{lccc}
\multicolumn{4}{c}{Ex-Post Model: South} \\ \hline
 & (1) & (2) & (3) \\
VARIABLES & Model 1 & Model 2 & Model 3 \\ \hline
 &  &  &  \\
post\_treatment & -0.009 & -0.057 & -0.061 \\
 & (0.014) & (0.045) & (0.055) \\
Constant & 0.103*** & 4.433*** & 4.132*** \\
 & (0.000) & (0.289) & (0.194) \\
 &  &  &  \\
Observations & 7,073 & 7,073 & 7,073 \\
 R-squared & 0.002 & 0.919 & 0.930 \\ \hline
\multicolumn{4}{c}{ Robust standard errors in parentheses} \\
\multicolumn{4}{c}{ *** p$<$0.01, ** p$<$0.05, * p$<$0.1} \\
\multicolumn{4}{c}{ See below.} \\
\end{tabular}

\end{table}
\begin{table}[h] %maybe put on top
    \centering
    \caption{Pull Factor Model: West}
    \label{tab: west_expost_model}
    \begin{tabular}{lccc}
\multicolumn{4}{c}{Ex-Post Model: West} \\ \hline
 & (1) & (2) & (3) \\
VARIABLES & Model 1 & Model 2 & Model 3 \\ \hline
 &  &  &  \\
Constant & 0.142*** & 3.864*** & 3.809*** \\
 & (0.000) & (0.180) & (0.158) \\
 &  &  &  \\
Observations & 5,185 & 5,185 & 5,185 \\
 R-squared & 0.002 & 0.913 & 0.919 \\ \hline
\multicolumn{4}{c}{ Robust standard errors in parentheses} \\
\multicolumn{4}{c}{ *** p$<$0.01, ** p$<$0.05, * p$<$0.1} \\
\end{tabular}

\end{table}
\begin{table}[h]
    \centering
    \caption{Pull Factor Model: Northeast}
    \label{tab: northeast_expost_model}
    \begin{tabular}{lccc}
\multicolumn{4}{c}{Ex-Post Model: Northeast} \\ \hline
 & (1) & (2) & (3) \\
VARIABLES & Model 1 & Model 2 & Model 3 \\ \hline
 &  &  &  \\
post\_treatment & 0.002 & -0.195*** & -0.190*** \\
 & (0.014) & (0.049) & (0.048) \\
Constant & 0.064*** & 3.873*** & 3.555*** \\
 & (0.000) & (0.107) & (0.200) \\
 &  &  &  \\
Observations & 3,011 & 3,011 & 3,011 \\
 R-squared & 0.001 & 0.936 & 0.940 \\ \hline
\multicolumn{4}{c}{ Robust standard errors in parentheses} \\
\multicolumn{4}{c}{ *** p$<$0.01, ** p$<$0.05, * p$<$0.1} \\
\end{tabular}

\end{table}
\begin{table}[h] %maybe put on top
    \centering
    \caption{Push Factor Model: West}
    \label{tab: west_exante_model}
    \begin{tabular}{lccc}
\multicolumn{4}{c}{Ex-Ante Model: West} \\ \hline
 & (1) & (2) & (3) \\
VARIABLES & Model 1 & Model 2 & Model 3 \\ \hline
 &  &  &  \\
ante\_treatment &  & 0.061 & 0.057 \\
 &  & (0.045) & (0.046) \\
Constant & 0.120*** & 4.092*** & 4.012*** \\
 & (0.000) & (0.127) & (0.156) \\
 &  &  &  \\
Observations & 5,044 & 6,807 & 6,807 \\
 R-squared & 0.002 & 0.934 & 0.937 \\ \hline
\multicolumn{4}{c}{ Robust standard errors in parentheses} \\
\multicolumn{4}{c}{ *** p$<$0.01, ** p$<$0.05, * p$<$0.1} \\
\end{tabular}

\end{table}
\FloatBarrier
\begin{table}[h] %try htbp in future test in other doc if this works elsewhere lol wow
    \centering
    \caption{Push Factor Model: Northeast}
    \label{tab: northeast_exante_model}
    \begin{tabular}{lccc} \hline
 & (1) & (2) & (3) \\
VARIABLES & Model 1 & Model 2 & Model 3 \\ \hline
 &  &  &  \\
ante\_treatment &  & -0.017 & 0.004 \\
 &  & (0.046) & (0.048) \\
Constant & 0.072*** & 4.167*** & 4.651*** \\
 & (0.000) & (0.159) & (0.367) \\
 &  &  &  \\
Observations & 3,220 & 4,684 & 4,684 \\
 R-squared & 0.001 & 0.917 & 0.924 \\ \hline
\multicolumn{4}{c}{ Robust standard errors in parentheses} \\
\multicolumn{4}{c}{ *** p$<$0.01, ** p$<$0.05, * p$<$0.1} \\
\end{tabular}

\end{table}

%MA
\clearpage
\newpage
\section{Massachusetts Figures}
\begin{figure}[h]
    \centering
    \includegraphics[width=0.75\linewidth, trim= 0 0 0 20, clip]{outputs/summary_stats/MA_post_trends.png}
    \caption{MA Pull Factor Model}
    \label{fig: MA_post_trends}
    \vspace{0.5em}
    \begin{minipage}{0.85\textwidth}
    \small \textit{Note:} There are data reliability concerns between 2005 and 2008 \citep{3, 4, 7}.
    \end{minipage}
\end{figure}

\begin{figure}[h]
    \centering
    \includegraphics[width=0.75\linewidth, trim= 0 0 0 20, clip]{outputs/summary_stats/MA_ante_trends.png}
    \caption{MA Push Factor Model}
    \label{fig: MA_ante_trends}
    \vspace{0.5em}
    \begin{minipage}{0.85\textwidth}
    \small \textit{Note:} There are data reliability concerns between 2005 and 2008 \citep{3, 4, 7}.
    \end{minipage}
\end{figure}

\end{document}
\begin{table}[htbp]

\caption{Summary Statistics by Region (1)}
\label{region_1} %added
\centering
\begin{tabular}[t]{rrrrr}
\toprule
\multicolumn{1}{c}{ } & \multicolumn{2}{c}{\% in the Midwest} & \multicolumn{2}{c}{\% in the South} \\
\cmidrule(l{3pt}r{3pt}){2-3} \cmidrule(l{3pt}r{3pt}){4-5}
Year & Same-Sex & Opposite-Sex & Same-Sex & Opposite-Sex\\
\midrule
2011 & 19.904 & 25.041 & 33.475 & 37.691\\
2012 & 19.891 & 25.036 & 33.792 & 37.746\\
2013 & 18.736 & 25.041 & 34.482 & 37.755\\
2014 & 20.224 & 24.922 & 34.505 & 37.845\\
2015 & 18.593 & 24.829 & 34.929 & 37.975\\
\addlinespace
2016 & 18.781 & 24.794 & 35.556 & 38.011\\
2017 & 19.101 & 24.664 & 35.537 & 38.047\\
2018 & 19.323 & 24.627 & 35.796 & 37.955\\
2019 & 18.446 & 24.437 & 36.848 & 38.101\\
\bottomrule
\end{tabular}
\vspace{0.5em}
\begin{minipage}{0.85\textwidth} % Adjust width as needed
\small \textbf{Note:} The Midwest includes Illinois, Indiana, Michigan, Ohio, Wisconsin, Iowa, Kansas, Minnesota, Missouri, Nebraska, North Dakota, and South Dakota. The South includes Delaware, District of Columbia, Florida, Georgia, Maryland, North Carolina, South Carolina, Virginia, West Virginia, Alabama, Kentucky, Mississippi, Tennessee, Arkansas, Louisiana, Oklahoma/Indian Territory, and Texas. These definitions are set by the ACS.
\end{minipage}
\end{table}
